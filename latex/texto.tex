% !TEX TS-program = pdflatex
% !TEX encoding = UTF-8 Unicode
\documentclass[12pt,a4paper,oneside]{book}



\usepackage{fancyhdr}
\usepackage[portuguese]{babel}
\usepackage{csquotes}

\title{\textbf {Biografia do Papa São Pio X}}
\author{Autoria: F.A. Forbes \\ Tradução: Mim}


\begin{document}
\setcounter{page}{3}



\maketitle

\pagebreak
\quad
\thispagestyle{fancy}
\pagenumbering{gobble}

\fancyfoot[R]{
    \footnotesize
``Havia dito a você que sou devoto de São Pio X''\\
-Papa Francisco, a Mons. Bonora, em 21 de agosto de 2015.\\
}


\pagebreak

\begin{center}
    {\section*{I. Garoto e estudante}}

\end{center}

 
\quad Na vila de Riese nas planícies venezianas, nasceu a 2 de junho de 1835, uma criança a qual estava destinada a deixar sua marca na história do mundo.

\quad Giuseppe Melchior Sarto¹ era o mais velho entre os oito filhos sobreviventes de Giovanni Battista Sarto, mensageiro municipal e carteiro de Riese, e de sua esposa Margheritta. Pessoas de renda humilde, o que às vezes dificultasse que conseguissem pagar suas contas. O dinheiro era labutado e escasso, e o futuro Papa se vestia, como disse um biógrafo italiano, ``como Deus quis''. Contudo, Giovanni Batista e sua esposa vieram de uma trabalhadora e piedosa estirpe, daqueles que resistem fortemente e sofrem pacientemente e que ensinaram os filhos a agir do mesmo modo.



O pequeno Bepi
 \footnote[1]{\emph{Beppo}, \emph{Beppino}, \emph{Bepi} e \emph{Beppe} todos são diminutivos do mesmo nome.
N.T. \emph{Sarto} do italiano, ``Alfaiate''.}
se destacava tanto por sua inteligência como por sua inquietude. O professor da vila, o qual o tinha como um broto merecedor de cultivo, foi --- nos é dito --- obrigado a utilizar de métodos mais persuasivos do que os aceitos para acalmar sua vivacidade. De fato, o ingrediente seráfico em Bepi
 parece ter sido bastante fermentado pela do pequeno humano****. ``Aquele moleque --- exclamou um senhor habitante de Riese ao saber da elevação do Cardeal Sarto ao papado --- muitas de minhas cerejas tiveram como fim o seu bucho! ''.


Não muito antes de Bepi ter aprendido os rudimentos da leitura e escrita, que era tudo o que uma escola de vilarejo poderia oferecer, o garoto tornara-se um competente coroinha na Missa. Tanta era sua influência sobre seus companheiros, que em seus 10 anos foi escolhido como líder de um tipo de grupeta dos acólitos que serviam na igreja da localidade. O pequeno mestre de cerimônias mostrou-se perfeitamente adequado à situação. Havia um bom e calmo temperamento e uma divertida sagacidade nos métodos do jovem Bepi
 que sua autoridade era irresistível e inquestionável.

\quad Para a maioria dos garotos que diariamente servem ao altar, a ideia de seguir uma vida sacerdotal, cedo ou tarde, por si mesma aparece; a alguns ela vem como um chamado encantadora. A vocação de Giuseppe parecia que tinha crescido consigo, de ter sido, desde a tenra idade, o centro de sua vida. A quase dois quilômetros de Riese, havia uma capela dedicada à Virgem Maria, nela havia uma estátua conhecida como a \emph{Madonna delle Cendrole}. Ali o jovem Bepi amava ir e rezar, deixando suas alegrias e tristezas ao pé da Mãe de Cristo, esta provavelmente foi a primeira confidente do seu desejo de consagrar sua vida a Deus. Certamente esse santuário era de especialmente caro a ele ao longo de sua vida, um local onde residia as mais alegres memórias de sua infância.

\quad Aos 12 anos de idade, o menino fez sua Primeira Comunhão. Será que ele pensou na demora que o tempo passara; foi ela, a memória de um desejo do seu infante coração, que o levou, anos mais tarde, a encurtar o tempo da espera das crianças de todo o mundo católico?

\quad Qualquer coisa que levasse ao conhecimento de Deus parecia possuir uma irresistível fascinação para Bepi. Nunca se foi dito de suas faltas às aulas do pároco \emph{Don} Tito Fusarini, e de seu vigário \emph{Don} Luigi Orazio, que ensinava a doutrina cristã às crianças da paróquia. Tão sagaz era sua inteligência e tão visível sua aptidão que \emph{Don} Luigi, que naquele momento ensinava latim ao seu próprio irmão mais novo, o tornou seu mais novo pupilo. O progresso do garoto logo convenceu seu tutor que ele possuía os germes de um estudioso e os dois padres se determinaram a prepará-lo para a escola de gramática em Castelfranco.

A uma distância de sete quilômetros de Riese, Castelfranco, com sua medieval e romântica atmosfera, sua antiga fortaleza e pitoresco mercado repleta de gente, não é a menos atrativa entre as antigas cidades venetas. Aqui, em 1447, nasceu Giorgione, aqui também na bela catedral encontra-se uma de sua mais belas \emph{Madonnas}. Em cada lado da Virgem Mãe, sentada no trono com seu Divino Filho aos braços, encontra-se São Francisco de Assis e São Liberal, patrono de Treviso, um jovem cavaleiro de armadura. \underline{(*colocar imagem?)}  
Várias devem ter sido as vezes que o jovem Giuseppe foi à quieta catedral rezar ante a \emph{Madonna}. Pedira ele a força de um guerreiro e a humildade de um frade para amar como o Cristo e ser puro como Sua Mãe? Aqueles que o conheceram [no tardar de sua vida] são testemunhas que (after-life?) essas graças lhe foram concedidas.

\quad Dia após dia, qualquer fosse a condição do tempo, o rapaz caminhava os sete quilômetros a Castelfranco   com seus sapatos pendiam sobre seus ombros, e um pedaço de pão ou um punhado de polenta em seu bolso. Ao quarto e último ano de seu período escolar de Giuseppe, juntou-se consigo seu irmão Angelo e dada a uma ligeira melhora nas condições financeiras de seu pai, os dois irmãos foram promovidos a condutores de uma velha carroça.

\quad O trabalho do dia ainda estava longe de terminar quando os rapazes chegavam da escola. Muito havia para ser feito dentro e fora de casa: a vaca e o burro tinham que ser cuidados, trabalhos no jardim e nos campos; Bepi se animava em ajudar sua mãe nos cuidados da casa e zelar pelos seus pequeninos irmãos e irmãs, logo sua mãe poderia tirar um merecido cochilo. Sua essência cativante e profunda generosidade o destacou entre os filhos, enquanto os mais novos o viam praticamente semelhante a um de seus pais.

\quad Em seu primeiro ano estudando em Castelfranco, Giuseppe Sarto se mostrou um pupilo brilhante e trabalhador, qualidade que não estão sempre acompanhadas. No final do quarto ano, nos exames que ocorreram no seminário diocesano de Treviso, ele ficou em primeiro em todas as disciplinas. Os dois sacerdotes de Riese estavam com razão orgulhosos do seu estudante e imaginavam grandes coisas no futuro. Educação, no entanto, custa dinheiro; e a família Sarto não só era pobre, mas possuíam oito filhos para criar. Que Bepi possui uma vocação ao sacerdócio era evidente a qualquer um que era próximo dele. O próximo passo era claramente o seminário, mas quem iria pagar as despesas? A côngrua de um pároco italiano não comporta tal investimento. Portanto, \emph{Don} Tito Fusarini foi ter com o Cônego Casagrande, prefeito dos estudos no seminário, o qual aplicou os exames aos garotos de Castelfranco, com certeza ele iria se interessar no brilhante jovem que obteve aprovação com honra em todas disciplinas.

\quad Aconteceu que o Patriarca de Veneza, Cardeal Jacopo Monico, era ele próprio filho de camponês e também cresceu naquela mesma vila de Riese. Conhecido não só pelo ser amor por cartas que por seu zelo pela religião, era ele o responsável pela bolsa de alguns seminaristas do seminário de Pádua. Que seu coração seria tocado pelo pensamento de um jovem conterrâneo o qual, também uma prole do povo, impedido de continuar sua educação religiosa por falta de meios, isso era uma hipótese muito provável. Logo, \emph{Don} Tito insistiu ao Cônego Casagrande, implorando que ele intercedesse pela causa de Giuseppe, pedido recebido com uma pronta e calorosa aceitação.

\quad Em Riese, dominava o suspense e a esperança. O carteiro era um homem de firme fé, cuja confiança em Deus nunca o frustrou e Margherita rezava sem cesar. Como o futuro de Bepi estava posto em jogo, a mais profundas esperanças de seu coração dependiam da resposta do Patriarca. Chegada a resposta, o Cônego Casagrande informou a \emph{Don} Fusarini que Giuseppe Sarto fora aceito como estudante no seminário de Pádua e que o Patriarca escreveu ele mesmo ao bispo local recomendando o jovem Sarto a seus próprios cuidados.

\quad Na alegria de Giuseppe, não havia lamento pelo pensamento de partir pela primeira vez sua humilde vila com todos os seus prezados conhecidos. Na névoa que pairava no início de uma manhã de Novembro, o rapaz de quinze anos juntos de seus poucos pertences guardados numa simples carroça, meio de transporte acessível aos pobres na época, segurando bravamente as lágrimas que dificilmente podiam ser reprimidas, despediu-se de sua família.

\quad Se o charme medieval de Castelfranco tinha influenciado tão profundamente o jovem estudante, havia tudo e mais um pouco na cidade de Pádua que satisfizesse o seu amor pela beleza. Famosa pelo mundo afora, a basílica de \emph{Il Santo} foi construída no século XIII*** e dedicada em honra ao grande Santo Antônio. Esculturas de Donatello, relevos dos irmãos Lombardo e quadros de Mantegna, Veronese e Giotto decoravam suas paredes. Já a catedral, parcialmente destruída no século XII, foi reconstruída por Michelangelo. A universidade, fundada no século XIII, tido em seu corpo discente homens como Vittorino da Feltre, o grande educador, e Giovanni da Ravenna, o amigo de Petrarca, foram famosos durante a Idade Média pelas suas escolas de medicina e direito.

\quad O seminário, fundado no ano de 1577 e bastante expandido nos século seguinte, comporta uma bela igreja e uma nobre biblioteca rica em preciosos manuscritos. Foi provavelmente a primeira biblioteca que Bepi viu e, certamente, a primeira na qual ele possuía liberdade --- quem pode imaginar o encanto de um jovem estudante ao vagar pelos seus grandes salões e realizar que seus tesouros iriam daqui em diante ser parte das bases de uma nova vida que agora era a sua.

\quad A inteligência e o animado bom humor de Giuseppe, junto com a sua maneira agradável de ser que parece o acompanhar desde a sua juventude, em pouco tempo o fez favorito entre seus colegas e mestres. Um destes escreveu uma vez para \emph{Don} Pietro Jacuzzi, que sucedeu \emph{Don} Orazio como vigário de Riese e muito amigo de Bepi ``Sua mente é sagaz, sua vontade forte e madura, sua produção extraordinária''. A disciplina um tanto rigorosa do seminário não apresentou dificuldade ao menino que em toda sua vida esteve acostumado com a abnegação. De fato, a sua disposta e inteligente submissão à autoridade foi uma característica marcante em sua vida. Antes de comandar --- iria dizer como Papa --- é necessário ter aprendido a obedecer.

\quad Ao fim de seu primeiro ano em Pádua, Giuseppe foi o aluno número um em todas as matérias. A vinda para a sua casa em Riese radiava alegria, tanto do rapaz quanto de sua família. Nas férias, passava o dia com os seus amigos de infância no campo que ele tanto amava. Para \emph{Don} Jacuzzi e \emph{Don} Fusarini, ele era um filho amado, de modo que boa parte de seu tempo passava no presbitério ou junto ao bom vigário. Embora fosse um período de descanso, não deixava de lado tempo para estudar. Outono e seus dias foram passando e logo terminaram.

\quad De volta a Pádua, Giuseppe iniciou seus afazeres com vigor, sem pressentir o sofrimento que em breve ofuscaria sua alegria. No mês de maio, seu pai veio a falecer após alguns dias doente, deixando sua esposa e uma grande família em circunstâncias penosas. O pensamento a respeito das dificuldades que sua mãe passava para enfrentar a pobreza pesou enormemente no coração de Giuseppe. Ele era o filho mais velho e deveria voltar para ajudá-la, porém a valorosa Margherita por nada iria permitir que o seu filho abandonasse seu caminho ao sacerdócio. Era uma mulher corajosa, além disso os outros irmãos estavam crescidos, logo estariam aptos a ajudar no sustento da família. Uma segunda dor seguiu-se da primeira. \emph{Don} Tito Fusarini, aquele que foi como um segundo pai a Bepi e cuja saúde ia mal o fez cada vez mais e mais se submeter a compaixão de seu vigário, até que finalmente teve que deixar sua ocupação em Riese.

\quad \emph{Don} Pietro Jacuzzi --- que o sucedeu como reitor --- era, desde o dia que chegou a vila, amigo próximo de Giuseppe e quem orientava o garoto em todas sua dificuldades próprias de um menino. O rapaz o tinha como modelo de tudo que o sacerdote deveria ser, continuamente trocava correspondências de Pádua. A ele devia o conhecimento e o amor pela música que se mostrou tão valiosa nos anos vindouros, não havia ele auxiliado a transformação que aconteceu no coro local sob a habilidosa direção de \emph{Don} Pietro? Foi também testemunha da generosidade e incansável devoção aos seus deveres sacerdotais com o qual obteve o amor e reverência de seus paroquianos; contudo, dentro de um ano, Giuseppe perdeu seu amigo novamente. \emph{Don} Pietro for transferido para Vascon, para a tristeza do povo de Riese.

\quad Quando Giuseppe vinha para casa durante as férias de outono de 1853, a profundida de sua perda se tornou claro a si. Riese dificilmente era Riese sem \emph{Don} Tito e \emph{Don} Pietro. O novo pároco, cujo caráter de pouca vivacidade contrastou fortemente com a amabilidade (rever adjetivo) marcante dos dois predecessores, não era popular. Não gostava de realizar visitas aos doentes a noite, (revisar) e o disse tranquilamente a seus paroquianos do púlpito. Porém a doença e a morte são boas em desconsiderar se são inoportunas ao pároco, ou mesmo a qualquer um, disso os habitante de Riese bem sabiam.

\quad Por sua posição como seminarista, Giuseppe tinha o dever de estar em uma situação amistosa com o presbítero. Por outro lado, no convívio com as pessoas da vila, não podia deixar de escutar críticas severas contra seu pastor. Embora fosse forçado a admitir a si mesmo que o métodos do recém-chegado padre eram um pouco singulares, a lealdade do menino e sua natureza reta não o permitia discutir sobre o assunto com seus amigos.
Nessa dificultosa e inconveniente situação, o rapaz de dezessete anos mostrou um tato e um discernimento que seria admirável mesmo em um homem com experiência, ``Essas férias tem sido um completo fracasso''--- escreveu a \emph{Don} Jacuzzi, que sabia por meio de outras cartas como as coisas iam --- ``Eu tento o quanto posso ficar em casa e tento quando visito meus familiares a deixar fora das conversas assuntos indesejáveis.''
\begin{quotation}
    \raggedleft{
        \small\emph{ ``Não há tormento mais doloroso \\
           que recordar, na miséria, o tempo venturoso;''\\}
       }    
\end{quotation}
\noindent
ele cita na mensagem, conhecia bem a obra de Dante. `` Até de cantar diminui. Espero pela meu pequeno quarto no seminário e a quieta vida de estudos''.

\quad Em 1856, Giuseppe se destacou mais do que nunca, faltavam apenas dois anos para concluir o seminário. Mesmo com um brilhante sucesso, mantinha-se modesto e humilde, enquanto animada bondade e simpatia o fez um forte influência para o bem entre seus jovens colegas. Tanto era a confiança que seus superiores lhe depositavam que durante um bom tempo foi prefeito de disciplina na sala geral de estudos. ``Meus mestres me chamam de '\emph{Giubilato}'\footnote[2]{Trocadilho entre \emph{Giuseppe} e \emph{giubilo}, isto é, \emph{júbilo} em italiano.}'', escreveu a \emph{Don} Pietro. ``Eu gostaria de poder demonstrar mais ainda minha gratidão pela bondade de meus mestres''. Ainda ele tinha muita estima pelo quarto particular preparado para ele durantes seus últimos dois anos em Pádua. ``Aqui leio e trabalho'' --- escreveu ao mesmo caro amigo --- ``e me preparo para a vida de solitude e estudo que será a minha como sacerdote''. Seus estudos favoritos eram a Bíblia e o Padres da Igreja. As cartas pastoras e encíclicas papais, anos mais tarde, testemunhariam o fato que sua predileção durou pela sua vida.

\quad Seu conhecimento e amor por música obteve para ele a direção do coro do seminário. ``Eu trabalhei tanto na música para festa de São Luís Gonzaga'' --- escreveu em junho de 1857 --- ``que estou realmente esgotado''.

\quad Em 27 de fevereiro do mesmo ano, ele foi ordenado subdiácono\footnote[3]{colocar algo sobre subdiácono} na catedral de Treviso e, na festa do Sagrado Coração, foi enviado a Riese para pregar. ``No último domingo, fui a Riese para realizar um pequeno discurso sobre o Sagrado Coração'', escreve para \emph{Don} Pietro. Ele não mencionou, porém, que seu pequeno discurso fora tão impressionante e tão eloquente que o entusiamo da congregação não conhecia limites.

\quad Ao final do mês de agosto de 1859, a vida de seminarista de Giuseppe Sarto acabava. Como possuía apenas 23 anos, e a lei canônica para ordenação era de 24 anos, o Bispo de Treviso escreveu para Roma para obter a dispensa. O jovem clérigo terminou seu último ano como fora o primeiro: com honras em todas as disciplinas. O registro de seu progresso triunfal ainda pode ser visto nos livros do seminário de Pádua, os professores uniram-se na parabenização pelas qualidades de seu caráter e não menos também pelas de sua inteligência. Em setembro a dispensa chegou, junto com o dia tanto desejado no qual Giuseppe Sarto seria para sempre consagrado para o serviço de Deus. O Bispo de Treviso estava então em Castelfranco, lá foi local onde a ordenação ocorreu.

\quad A névoa do outono repousava como um véu sobre a paisagem familiar enquanto o jovem rapaz se deslocava pela via que levava de Riese a Castelfranco. A cavalo trotava ligeiramente, assim mesmo o caminho nunca antes parecera ser tão longo. Tantas eram as vezes que ele vagava no passado pela poeira, lama e neve, a pé descalço para proteger seus sapatos, valioso fruto do trabalho da família Sarto. E foi o pensamento do dia que enfim tinha amanhecido, um dia que parecia tão longínquo e tão impossível, que foi a inspiração e a força daquela vida de dificuldades, que fez tudo ser mais fácil de suportar. A alegria imensa que agora o possuía sobrepunha-se diante o passado. O primeiro relance das paredes cobertas de hera de Castelfranco fez seu coração bater ao ponto de quase o sufocar. ``Hoje serei um sacerdote'' era o único pensamento em sua mente. Um pouco depois, no momento em que ele se ajoelhou diante do altar da catedral, onde ele frequentemente rezava quando criança, para receber a sagrada imposição das mãos, pareceu para ele que o mundo não tinha mais nada para lhe oferecer.

\quad No dia seguinte, o neo-sacerdote celebrou sua primeira Missa na igreja paroquial de Riese. Quem pode descrever a felicidade de sua mãe ao ouvir aquela amada voz, clara e ressonante que ainda se fazia presente quando mais velho, embora trêmula pela alegria e temor to momento, pronunciando as palavras de tão grande Mistério? A Missa terminou, a congregação se reuniu para beijar as mãos do jovem padre que conheciam e amavam desde sua infâncias --- mãos que no dia tocaram pela primeira vez o Corpo do Senhor. Para dizer que foi um dia festivo em Riese, embora pouco expressa o júbilo geral.

\quad Alguns dias depos, \emph{Don} Giuseppe recebeu uma carta anunciando seu destino. O Bispo de Treviso o apontou como vigário de \emph{Don} Antonio Constantini, pároco de Tombolo.


\pagebreak

\begin{center}
    {\section*{II. Vigário e pároco}}

\end{center}


\quad  A vila de Tombolo, situada na diocese de Treviso em Pádua, é cercada por colinas e florestas e abastecida pelas águas afluentes do rio Brenta. A igreja paroquial, dedicada a Santo André Apóstolo*, tinha sua presença no centro da pequena localidade. Não possuía indústrias, uma comuna orgulhosamente campeira. Grande parte da população ocupava-se do cultivo da terra e na criação de gado. Um povo possuidor de características marcantes: robusto, resistentes ao sol, chuva e vento, voz rouca e um jeito não muito lá elegante, não obstante, tinham um coração grande e piedosos à sua maneira.

\quad Contudo os habitantes tinham um vício --- ou tiveram quando \emph{Don} Giuseppe tornou-se seu vigário. Eles blasfemavam \footnote[3]{N.T. Aqui acredita-se que o autor esteja se referindo ao costume de se utilizar palavrões envolvendo Deus e outros termos santos.} em vão regularmente e aos montes** a qualquer coisa, entre si e aos quatro ventos. ``Claro, sem querer ofender o Senhor Deus,'' --- tentavam ingenuamente explicar ao jovem Padre --- ``Com certeza Ele entende. Tente ir ao mercado vender seus animais ou sua colheita com um 'por favor' e um 'obrigado' e verás o que irás receber!''

\quad O pároco, \emph{Don} Antonio Constatini, frequentemente se encontrava doente. Devoto ao seu rebanho e totalmente desejoso do melhor aos seus, sua pouca saúde era um constante impedimento. Os dois compartilhavam vários gostos, notavelmente o amor pela música e os estudos bíblicos e patrísticos. Ele logo começou a ver \emph{Don} Giuseppe como um filho e apreciava muito as suas boas qualidades.

\quad `` Eles me enviaram um jovem como vigário,'' --- escreveu a um amigo --- ``com ordens de o formar para seus deveres como pároco. Eu te asseguro que provavelmente será o contrário. Tão zeloso, tão sensato e outros preciosos dons que muito acho que terei o que aprender com ele \emph{(consigo?)}. Algum dia ele deverá portar a mitra, disto estou certo, e depois? Quem sabe?  ''

\quad Nem por isso o pároco não deu seu melhor para cumprir com sua missão. ``\emph{Don} Bepi,'' --- dizia a seu jovem vigário algo desse tipo --- ``não gostei disso ou daquilo em teu último sermão''. Quando a igreja estava vazia, colocava \emph{Don} Bepi no púlpito para pregar, criticando tanto o conteúdo como o método utilizados; comentários valiosos, na verdade, pois \emph{Don} Antonio era um homem de muito estudo e um excelente teólogo. Diante disso, \emph{Don} Bepi, cujos sermões já estavam começando a ficar famosos pelos campos por seu zelo e eloquência, os ouvia humildemente e prometia melhorar na próxima.

\quad O que o jovem vigário recebia era quase nada, já que Tombolo era uma paróquia bem pobre, mas ele não estava acostumado com o luxo. Havia planejado sua vida de sacerdote antes da ordenação, um plano que se mostrava difícil de se cumprir. Consistia em estudar muito para tornar-se mais apto à pregação, fazer o bem o quanto fosse possível no confessionário e ao púlpito, auxiliar seu rebanho tanto material como moralmente, visitar os doentes, socorrer o pobre e instruir os ignorantes --- assim era seu projeto; e, com todo o vigor de sua alma, se entregava ao trabalho.

\quad \emph{Don} Antonio tinha uma sobrinha, já viúva, a qual cuidava da casa para seu tio, costumava ver uma luz acesa na janela de \emph{Don} Giuseppe. Era a última coisa que via a noite e a primeira ao amanhecer.

``O senhor nunca vai para cama, \emph{Don} Bepi?'' --- Ela lhe perguntou uma vez no café da manhã, enquanto o vigário tomava sua refeição na casa paroquial.

\emph{Don} Bepi riu. ``Eu estudo um bocado'' --- respondeu. Ele depois confessou que dormia por quatro horas e achava que lhe era o suficiente.

\quad ``Ele era magro como uma vassoura'' --- dizia a boa senhora quando entrevistada perguntada sobre suas memórias --- ``ele comia apenas o necessário para manter seu corpo junto à alma, e nunca estava fraco. (never off his feet)''.

\quad De manhã, frequentemente era ele quem tocava o sino da igreja para a Missa para não incomodar o sacristão. Então, ia chamar \emph{Don} Antonio, tendo já preparado o necessário para ele. Às vezes acontecia de achar o senhor Padre indisposto e sem condições de se levantar.

\quad ``Qual o problema?'' --- ele lhe perguntava em seu modo alegre --- ``mais uma noite mal dormida?''

\quad ``Receio não conseguir me levantar'', lamentava \emph{Don} Antonio.

\quad ``Ficais calmo, será melhor que continues na cama. Podes ficar tranquilo que irei cuidar de tudo.'', continuava a alegre voz.

\quad ``Mas tu já possuis um sermão para pregares hoje, meu caro Bepi''.

\quad ``Qual o problema? Pregarei dois''.

\quad Mesmo nesses dias de recuperação de \emph{Don} Antonio, além do trabalho dobrado, \emph{Don} Giuseppe cuidava do pobre pároco. Como ele manejava tudo isso, só ele o sabia.

\quad Não tinha esquecido --- nem tinha como ele esquecer --- o linguajar deplorável de seus paroquianos. O vigário procurava o quanto podia interagir com as pessoas, fazendo amizade principalmente com os rapazes e garotos. Inteirava-se a respeito de seus trabalhos e participava de seus lazeres, tratava-os com espírito amigável de camarada de modo que quando aparecia, vários vinham a seu encontro. Num dia, um deles magoado disse ao sacerdote que não sabia ler nem escrever.

\quad ``Façamos, portanto, aulas à noite'' --- propôs \emph{Don} Bepi --- ``e irei lhes ensinar''.

\quad ``Seria muito complicado' --- um deles objetou ---``alguns de nós sabem um pouco, outros menos e outros ainda nada''.

\quad ``Qual o problema?'', respondeu o Padre. ``Teremos duas turmas: para aqueles que sabem algo e para aqueles que nada sabem. O professor da escola ensinará os que estão à frente e eu ensinarei o alfabeto''.

\quad ``Por que esse professor não ensina o alfabeto?'', interveio um fiel admirador de \emph{Don} Giuseppe.

\quad O Padre riu e disse: ``O alfabeto que é difícil, prefiro cuidar disso''.

\quad  ``Mas não podemos tomar seu tempo por nada'', declarou um. ``Como podemos retribuir ao senhor?''

\quad ``Parem de xingar,'' --- prontamente respondeu Bepi --- ``e serei bem mais que recompensado''.

\quad A escola de canto fez um rápido progresso em suas mãos.\emph{Don} Antonio, que como seu vigário, era um amante ardente de canto gregoriano, calorosamente o ajudava em todos seus esforços. A poderosa e destoante voz do coro da vila tornou-se quieta e orante sob sua direção. Se algum de seus acólitos mostrasse sinais de vocação ao sacerdócio, \emph{Don} Giuseppe dava aulas privadas até que soubesse o suficiente para o exame no seminário diocesano.

\quad Em um ponto, porém, \emph{Don} Antonio e seu vigário nunca concordavam. Qualquer quantia que sobrava da pequena renda de \emph{Don} Giuseppe era diretamente destinada aos pobres. Estes sabiam, assim quando ia pregar às vilas vizinhas, esperavam pelo recebimento do modesto punhado de seu bolso. Acontecia que às vezes ele chegava à casa sem um tostão, e \emph{Don} Antonio o admoestava.

\quad ``Não é justo com sua mãe, Bepi'', ele dizia --- ``devias pensar nela.''

\quad ``Deus proverá à minha mãe'', respondia. ``Essas pobres almas precisavam mais do que ela.''

\quad Convites para pregar em outras paróquias começaram a ficar mais frequentes. O que ele falava era sempre simples, porém era cheio de ensinamentos e iam direto aos corações. O jovem padre, além disso, uma eloquência natural e uma bela e sonora voz. Era tão evidente que ele falava com a totalidade de uma alma acesa com o Amor de Deus que seu entusiasmo era envolvente e seus sermões frutuosos. Houve uma ocasião em que o padre foi convidado para pregar em dia de festa na vila 


































\end{document}